% vim: ts=4 sts=4 sw=4 et tw=75
% preamble here.

\documentclass[nofonts,oneside,fancyhdr]{ctexbook}

\usepackage{geometry}
\usepackage{fontspec}
\usepackage{xeCJK}
\usepackage{amssymb}
\usepackage{hyperref}
% for number of footnote
\usepackage{pifont}
% 一页结束时, 脚注编号清零
\usepackage[perpage]{footmisc}
% verbatim and endverbatim
\usepackage{graphicx}
\usepackage{listings}
\usepackage{multicol}
\usepackage{tocloft}
\usepackage{fancyhdr}

% for bracketframe environment
\usepackage[tikz]{bclogo,rotating}
\usepackage{tikz}
\usepackage{mdframed}
\usepackage{minted}

% for fixing "Unable to load picture or PDF file 'bc-book.mps'"
\DeclareGraphicsRule{.mps}{eps}{*}{}

% 脚注编号带圈
\renewcommand\thefootnote{\ding{\numexpr171+\value{footnote}}}

% from package geometry
% 为边注加边框
\let\oldmarginpar=\marginpar
\renewcommand\marginpar[1]{%
    \oldmarginpar{\framebox{#1}}%
}
\geometry{%
    margin=1cm,
    marginparsep = 0.5cm,
    marginparwidth=1cm,
    top = 2.5cm,
    bottom = 2cm,
    outer = 1.8cm,
    inner = 1.8cm
}

% 页面风格
\pagestyle{fancy}
\fancyhead[LE,RO]{\rightmark}
\fancyhead[LO,RE]{\leftmark}
\fancyfoot[C]{\thepage}

% from package hyperref
\hypersetup{
    bookmarksnumbered = true,
    pdftitle = {Hacking Vim 7.2},
    pdfcreator = {wuzhouhui250@gmail.com},
    pdfauthor = {Kim Schulz},
    pdfsubject = {Vim},
    colorlinks = false,
    pdfborder = 0 0 0,
    pdfkeywords = {vim, text editing}
}


\setCJKfamilyfont{heiti}{FandolHei}
% 章节格式
\CTEXsetup[format={\CJKfamily{heiti}\large\upshape}]{subsection}
\CTEXsetup[numberformat={\CJKfamily{heiti}\large\upshape\bfseries}]{subsection}
\CTEXsetup[format={\CJKfamily{heiti}}]{subsubsection}
% 目录格式
\renewcommand\cftchapfont{\CJKfamily{heiti}}
\settowidth\cftchapnumwidth{第几十几章} % 最宽的可能编号
\renewcommand\cftchapaftersnumb{\hspace{2.2em}} % 额外间距

% from package fontspec and xeCJK
\setCJKmainfont{AR PL UMing CN}
\setCJKsansfont{AR PL UMing CN}
\setCJKmonofont[Scale=0.9]{AR PL UMing CN}
\setmainfont{FreeSerif}
\setsansfont{FreeSans}
% "Mapping={}" make quote symbol straight
\setmonofont[Mapping={}]{FreeMono}

\newcommand\vi{\texttt{vi}}
\newcommand\vim{\texttt{vim}}

\usetikzlibrary{calc}
\newenvironment{bracketframe}[1]
{\par\medskip\noindent
\begin{tikzpicture}
  \node[inner sep = 0pt] (box) \bgroup%
  \begin{minipage}[t]{44em}
    \begin{minipage}{6em}
        \centering
        \includegraphics[scale=1.8]{#1}
    \end{minipage}%
    \begin{minipage}{36em}
    \par\smallskip
    \surroundwithmdframed
      [topline=false, bottomline=false, leftline=false, rightline=false,
      backgroundcolor=lbcolor]
      {minted}}% former part
{%
    \end{minipage}\hfill
  \end{minipage}%
  \egroup;
  \draw[black,line width=3pt]
    ( $ (box.north east) + (-5pt,3pt) $ ) -- ( $ (box.north east) + (0,3pt) $ )
        -- ( $ (box.south east) + (0,-3pt) $ ) -- + (-5pt,0);
  \draw[black,line width=3pt]
    ( $ (box.north west) + (5pt,3pt) $ ) -- ( $ (box.north west) + (0,3pt) $ )
        -- ( $ (box.south west) + (0,-3pt) $ ) -- + (5pt,0);
\end{tikzpicture}
\par\medskip}

\newenvironment{warning}%
{
    \begin{center}
    \begin{minipage}{44em}
    \begin{bracketframe}{./images/bc-book.mps}
}
{
    \end{bracketframe}
    \end{minipage}
    \end{center}
}

\newenvironment{tips}%
{
    \begin{center}
    \begin{minipage}{44em}
    \begin{bracketframe}{./images/bc-lampe.mps}
}
{
    \end{bracketframe}
    \end{minipage}
    \end{center}
}

\newcommand\file[1]{\texttt{#1}}
\newcommand\newterm[1]{#1}
\newcommand\email[1]{\href{mailto:#1}{#1}}
\newcommand\key[1]{\textit{#1}}

\newenvironment{vimcmdform}
    {\list{}{\leftmargin=2em\rightmargin=0em}\item[]}
    {\endlist}

\lstnewenvironment{vimcode}
    {\lstset{basicstyle=\ttfamily,
        xleftmargin=2em,
        breaklines=true,
        tabsize=4}}
    {}

\title{深入 Vim 7.2}
\author{Kim Schulz \and \url{https://github.com/wuzhouhui/hacking_vim} }
